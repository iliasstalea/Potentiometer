%%%%%%%%%%%%%%%%%%%%%%%%%%%%%%%%%%%%%%%%%%%%%%%%%%%%%%%%%%%%%%%
%
% Seminar: Potentiometer 
% Iliass Talea
%
%%%%%%%%%%%%%%%%%%%%%%%%%%%%%%%%%%%%%%%%%%%%%%%%%%%%%%%%%%%%%%%
\documentclass[aspectratio=169]{beamer}

% --- THEMA UND FARBEN ---
\usetheme{Madrid}
\usecolortheme{beaver}

% --- PAKETE ---
\usepackage[utf8]{inputenc}
\usepackage[T1]{fontenc}
\usepackage[german]{babel}
\usepackage{amsmath}
\usepackage{graphicx}
\usepackage{booktabs}
\usepackage{siunitx} % Für korrekte Darstellung von Einheiten
\usepackage{tikz}
\usetikzlibrary{circuits.ee.IEC, positioning, arrows.meta}
\usepackage{microtype} % Verbessert die Typografie
\usepackage{hyperref} % Für klickbare Links im Quellenverzeichnis
\hypersetup{colorlinks=true, urlcolor=blue, linkcolor=black}
% --- ADD THIS LINE TO TURN ON CAPTION NUMBERING --
\setbeamertemplate{caption}[numbered]

% --- TITELINFORMATIONEN ---
\title[Potentiometer]{Seminar: Das Potentiometer}
\subtitle{Analyse eines elektromechanischen Transducers}
\author{Iliass Talea}
\institute{THGA}
\date{\today}

% --- FUSSZEILE ANPASSEN (KORRIGIERTE VERSION) ---

\setbeamertemplate{footline}
{
  \leavevmode%
  \hbox{%
  \begin{beamercolorbox}[wd=.4\paperwidth,ht=2.25ex,dp=1ex,left]{author in head/foot}%
    \hspace{2ex}\usebeamerfont{author in head/foot}\insertshortauthor
  \end{beamercolorbox}%
  \begin{beamercolorbox}[wd=.2\paperwidth,ht=2.25ex,dp=1ex,center]{title in head/foot}%
    \usebeamerfont{title in head/foot}\insertshorttitle
  \end{beamercolorbox}%
  \begin{beamercolorbox}[wd=.4\paperwidth,ht=2.25ex,dp=1ex,right]{date in head/foot}%
    \usebeamerfont{date in head/foot}\insertshortdate{}\hspace*{2em}%
    \insertframenumber{} / \inserttotalframenumber\hspace*{1ex}%
    % Logo ist jetzt HIER integriert, um Layout-Fehler zu vermeiden
    \includegraphics[height=1.8ex]{THGA-Logo.png}\hspace*{2ex}
  \end{beamercolorbox}}%
  \vskip0pt%
}


\usepackage{listings}
\usepackage{xcolor}

\lstdefinelanguage{json}{
    basicstyle=\ttfamily\footnotesize,
    numbers=left,
    numberstyle=\tiny,
    stepnumber=1,
    numbersep=5pt,
    showstringspaces=false,
    breaklines=true,
    frame=single,
    backgroundcolor=\color{gray!10},
    literate=
     *{0}{{{\color{blue}0}}}{1}
      {1}{{{\color{blue}1}}}{1}
      {2}{{{\color{blue}2}}}{1}
      {3}{{{\color{blue}3}}}{1}
      {4}{{{\color{blue}4}}}{1}
      {5}{{{\color{blue}5}}}{1}
      {6}{{{\color{blue}6}}}{1}
      {7}{{{\color{blue}7}}}{1}
      {8}{{{\color{blue}8}}}{1}
      {9}{{{\color{blue}9}}}{1}
      {:}{{{\color{red}{:}}}}{1}
      {,}{{{\color{red}{,}}}}{1}
      {\{}{{{\color{orange}{\{}}}}{1}
      {\}}{{{\color{orange}{\}}}}}{1}
      {[}{{{\color{orange}{[}}}}{1}
      {]}{{{\color{orange}{]}}}}{1}
      {"}{{{\color{green}{"}}}}{1}
}














\begin{document}

% --- TITELFOLIE (angepasst, professionell im Seminar-Design) ---
{
  \setbeamercolor{background canvas}{bg=white} % weißer Hintergrund
  \begin{frame}[plain]

    % Hintergrundbalken
    \begin{tikzpicture}[remember picture,overlay]
      \fill[red!70!black] (current page.north west) rectangle ([yshift=-3cm]current page.north east);
    \end{tikzpicture}

    % Titel oben im Balken (weiß)
    \vspace*{0.3cm}
    {\color{white}\Huge \textbf{Seminar: Das Potentiometer}} \\[0.3cm]
    {\color{white}\large Analyse eines elektromechanischen Transducers}

    \vfill

    % Autor / Uni unten links
    \begin{flushleft}
      \textbf{Iliass Talea} \\ 
      Technische Hochschule Georg Agricola (THGA) \\ 
      Seminarleiter: Prof. Dr.-Ing. Björn Keune \\ 
      \today
    \end{flushleft}

    % Logo unten rechts
    \begin{flushright}
      \includegraphics[height=1.5cm]{THGA-Logo.png}
    \end{flushright}

  \end{frame}
}

% --- INHALTSVERZEICHNIS ---
\begin{frame}{Agenda}
  \tableofcontents
\end{frame}

%%%%%%%%%%%%%%%%%%%%%%%%%%%%%%%%%%%%%%%%%%%%%%%%%%%%%%%%%%%%%%%
\section{1. Einleitung und Klassifizierung}
%%%%%%%%%%%%%%%%%%%%%%%%%%%%%%%%%%%%%%%%%%%%%%%%%%%%%%%%%%%%%%%

\begin{frame}{Das Potentiometer als elektromechanischer Transducer}
    
    
    \begin{columns}[c]
        
        \begin{column}{0.45\textwidth}
            \begin{figure}
                
                \includegraphics[width=\textwidth, height=5cm, keepaspectratio]{aufbaupoti.jpg}
                \caption{Prinzipieller Aufbau eines Drehpotis.}
            \end{figure}
        \end{column}

        \begin{column}{0.55\textwidth}
            \begin{block}{ Definition}
                Ein Potentiometer ist ein passiver, elektromechanischer Wandler, der eine mechanische Position in eine proportionale Widerstandsänderung umwandelt.
            \end{block}
            
            \medskip 
            
            \textbf{Kernkomponenten:}
            \begin{itemize}
                \item Stationäres Widerstandselement
                \item Beweglicher Schleiferkontakt
                \item Drei elektrische Anschlüsse
            \end{itemize}
        \end{column}

    \end{columns}
\end{frame}



% ---------------- Folie 1: Tabelle ----------------
\begin{frame}{Klassifikation von Potentiometern}
\centering
\begin{tabular}{lll}
\toprule
\textbf{Kategorie} & \textbf{Beispiele} & \textbf{Hinweis / Farbe} \\
\midrule
Bauformen & Drehpoti, Schiebepoti, Trimmer & \colorbox{blue!15}{\strut Form} \\
Materialien & Leitplastik, Cermet, Drahtgewickelt & \colorbox{green!15}{\strut Material} \\
Funktionsprinzip & Mechanisch, Digitalpoti, Hall-basiert & \colorbox{orange!20}{\strut Prinzip} \\
\bottomrule
\end{tabular}

\vspace{0.7cm}

% ---------------- Professioneller Rahmen ----------------
\begin{tikzpicture}
\node[
    fill=blue!20,            % Hintergrundfarbe
    text=blue!900,           % Textfarbe
    rounded corners=2mm,     % kleine Abrundung
    font=\footnotesize\bfseries,
    minimum height=0.8cm,
    minimum width=5cm,
    align=center,
    path picture={
        % optional: kleines Icon links einfügen
        % \node at (path picture bounding box.west) {\includegraphics[height=0.6cm]{icon.png}};
    }
] at (0,0) {Fokus des Seminars: Leitplastik- und Drahtpotentiometer};
\end{tikzpicture}


\end{frame}

% ---------------- Folie 2: Bild ----------------
\begin{frame}{Leitplastik- und Drahtpotentiometer}
    \begin{columns}[c]

        % --- Linkes Bild ---
        \begin{column}{0.48\textwidth}
            \begin{figure}
                \centering
                \includegraphics[width=0.7\linewidth]{bauformen.jpg} 
                \caption{Bauformen: Drehen, Schieben, Trimmen.}
                \label{fig:bauformen}
            \end{figure}
        \end{column}

        % --- Rechtes Bild ---
        \begin{column}{0.48\textwidth}
            \begin{figure}
                \centering
                \includegraphics[width=0.7\linewidth]{funktion.jpg} % kleineres Bild
                \caption{Funktionsprinzip: Mechanisch, Digitalpoti, Hall-basiert.}
                \label{fig:funktion}
            \end{figure}
        \end{column}

    \end{columns}
\end{frame}





%%%%%%%%%%%%%%%%%%%%%%%%%%%%%%%%%%%%%%%%%%%%%%%%%%%%%%%%%%%%%%%
\section{2. Physikalische Realisierung}
%%%%%%%%%%%%%%%%%%%%%%%%%%%%%%%%%%%%%%%%%%%%%%%%%%%%%%%%%%%%%%%

\begin{frame}{Mechanische Potentiometer im Überblick}
    \begin{columns}[T]
        \begin{column}{0.55\textwidth}
            \begin{block}{Drahtwickel-Potentiometer}
                \begin{itemize}
                    \item Widerstand aus fein gewickeltem Draht
                    \item Hohe Linearität und Belastbarkeit
                    \item Endliche Auflösung durch Drahtwicklungen
                    \item Ideal für präzise Messgeräte und industrielle Sensorik
                \end{itemize}
            \end{block}

            \begin{block}{Leitplastik-Potentiometer}
                \begin{itemize}
                    \item Widerstandsbahn aus leitfähigem Kunststoff
                    \item Kostengünstig und quasi unendliche Auflösung
                    \item Geringeres Kontaktrauschen
                    \item Typisch in Audio- und Konsumelektronik
                \end{itemize}
            \end{block}
        \end{column}
        \begin{column}{0.45\textwidth}
            \begin{figure}
                % <<< THE FIX IS HERE
                % We limit the height to 6cm and keep the aspect ratio.
                \includegraphics[width=\textwidth, height=5.5cm, keepaspectratio]{plastik.png}
                
                \caption{\textcolor{black}{Drahtwickel- (links) vs. Leitplastik-Potentiometer (rechts).}}
            \end{figure}
        \end{column}
    \end{columns}
\end{frame}



\begin{frame}{Mechanische Potentiometer im Überblick}
  \begin{center}
    \href{https://drive.google.com/file/d/1p4Oiq_yHVEMuO02oNpYVuEO2o35WG-JK/view?usp=drive_link}{
      \includegraphics[width=0.5\linewidth]{video.png}
    }

    \vspace{0.1cm}
    \small{Klicken Sie auf das Bild, um das Video abzuspielen.}
  \end{center}
\end{frame}


%%%%%%%%%%%%%%%%%%%%%%%%%%%%%%%%%%%%%%%%%%%%%%%%%%%%%%%%%%%%%%%
\section{3. Analyse der elektrischen Eigenschaften}
%%%%%%%%%%%%%%%%%%%%%%%%%%%%%%%%%%%%%%%%%%%%%%%%%%%%%%%%%%%%%%%

\begin{frame}{Der ideale Spannungsteiler vs. Realität}
    Im Idealfall (unbelasteter Ausgang) ist die Ausgangsspannung eines linearen Potentiometers exakt proportional zur Position des Schleifers:
    \[ U_{out}(a) = U_{in} \cdot a \quad \text{mit} \quad a \in [0, 1] \]
    wobei $a$ die normalisierte Schleiferposition ist.

    \pause

    \begin{beamercolorbox}[sep=1ex,center,wd=\textwidth]{block title}
        Die Realität ist jedoch komplexer!
    \end{beamercolorbox}
\end{frame}

\begin{frame}{Analyse der Realität: Störfaktoren}
    \begin{alertblock}{In der Praxis wird die ideale Übertragungsfunktion durch mehrere Faktoren gestört:}
    \begin{itemize}
        \item Der \textbf{Belastungseffekt} durch den Lastwiderstand
        \item Die \textbf{endliche Auflösung} (bei Drahtwickel-Potis)
        \item Das \textbf{Kontaktwiderstandsrauschen (CRV)}
        \item \textbf{Parasitäre Effekte} bei hohen Frequenzen
    \end{itemize}
    \end{alertblock}
\end{frame}

\begin{frame}{Kritischer Faktor: Der Belastungseffekt}
    \textbf{Problem:} Der an den Schleifer angeschlossene Lastwiderstand $R_L$ bildet eine Parallelschaltung mit dem Teilwiderstand $R_2$.

    \begin{columns}[c]
        \begin{column}{0.55\textwidth}
            Dies führt zu einer \textbf{nichtlinearen} Übertragungsfunktion.
            \begin{block}{Spannungsteilerformel (belastet)}
              \[ U_{out} = U_{in} \cdot \frac{R_2 || R_L}{R_1 + (R_2 || R_L)} \]
            \end{block}
            \vspace{0.2cm}
            Mit $R_2||R_L = \tfrac{R_2 R_L}{R_2+R_L}$. Der Fehler ist maximal, wenn $R_L$ in der Größenordnung des Potentiometerwiderstands liegt.
        \end{column}
        \begin{column}{0.45\textwidth}
            \begin{tikzpicture}[circuit ee IEC, thick, scale=0.9]
              \draw (0,0) to[resistor={info=$R_1$}] (0,-2)
                          to[resistor={info=$R_2$}] (0,-4) node[ground]{};
              \draw (0,-2) -- (2,-2)
                          to[resistor={info=$R_L$}] (2,-4) node[ground]{};
              \node at (-0.5,0) {$U_{in}$};
              \node at (2.5,-2) {$U_{out}$};
            \end{tikzpicture}
        \end{column}
    \end{columns}
\end{frame}

%%%%%%%%%%%%%%%%%%%%%%%%%%%%%%%%%%%%%%%%%%%%%%%%%%%%%%%%%%%%%%%
\section{4. Erweiterte Betrachtungen}
%%%%%%%%%%%%%%%%%%%%%%%%%%%%%%%%%%%%%%%%%%%%%%%%%%%%%%%%%%%%%%%

\begin{frame}{Kontaktwiderstandsrauschen (CRV)}
    \textbf{Definition:} Fluktuationen im Übergangswiderstand zwischen Schleifer und Widerstandsbahn, die eine Rauschspannung erzeugen. \\
    
    \textbf{Mathematisches Modell:}
    \[ U_{noise}(t) \approx k \cdot I(t) \cdot \Delta R_c(t) \]
    mit $\Delta R_c$ = zeitabhängige Schwankung des Kontaktwiderstands.

    \begin{exampleblock}{Industrielle Relevanz}
        In Audioverstärkern führt CRV zu hörbarem „Kratzen“ beim Verstellen der Lautstärke. Hochwertige Leitplastik-Potis minimieren diesen Effekt.
    \end{exampleblock}
\end{frame}

\begin{frame}{Parasitäre Induktivität und Kapazität}
    Drahtgewickelte Potentiometer zeigen parasitäre Induktivitäten $L_p$, während großflächige Bahnen parasitäre Kapazitäten $C_p$ erzeugen.

    \begin{block}{Ersatzschaltbild bei hohen Frequenzen}
    \[ Z_{eff}(\omega) = R + j\omega L_p + \frac{1}{j\omega C_p} \]
    \end{block}

    \begin{itemize}
        \item Bedeutend bei HF-Anwendungen (z. B. Radar, Kommunikation).
        \item Lösung: Einsatz von \textbf{speziellen HF-Potis} oder \textbf{digitale Alternativen}.
    \end{itemize}
\end{frame}

%%%%%%%%%%%%%%%%%%%%%%%%%%%%%%%%%%%%%%%%%%%%%%%%%%%%%%%%%%%%%%%
\section{5. Industrielle Anwendungen}
%%%%%%%%%%%%%%%%%%%%%%%%%%%%%%%%%%%%%%%%%%%%%%%%%%%%%%%%%%%%%%%

\begin{frame}{Anwendungen: Luft- und Raumfahrt}
    \begin{columns}[c] 

      % --- COLUMN 1: TEXT ---
      \begin{column}{0.5\textwidth}
        \begin{itemize}
          \item Positionssensoren in \textbf{Fly-by-Wire}-Systemen (z.B. Joystick-Abfrage).
          \item Steuerung der Ausrichtung von \textbf{Satellitenantennen}.
          \item Hohe Robustheit gegen Strahlungseinflüsse (besonders drahtgewickelte Varianten).
        \end{itemize}
      \end{column}

      % --- COLUMN 2: IMAGE ---
      \begin{column}{0.5\textwidth}
        \begin{figure}
            \centering
            \includegraphics[width=150]{potisaerospace.jpg}
            
            
            
            \vspace{0cm} 
            
           
            \caption{Hochpräzisions-Potentiometer für die Luftfahrtsteuerung.}
        \end{figure}
      \end{column}

    \end{columns}
\end{frame}

\begin{frame}{Anwendungen: Robotik und Medizintechnik}
    \begin{block}{Robotik}
      Gelenkwinkelmessung in Industrierobotern: Eine robuste und kostengünstige Methode zur Positionsrückführung in rauen Umgebungen.
    \end{block}
    \begin{block}{Medizintechnik}
      Exakte Positionierung in Infusionspumpen oder die Einstellung von Parametern an Dialysegeräten.
    \end{block}
    \vspace{0.3cm}
    Alternative: Kontaktlose \textbf{Hall-Sensoren} gewinnen an Bedeutung, sind jedoch oft teurer und temperaturabhängiger.
\end{frame}

%%%%%%%%%%%%%%%%%%%%%%%%%%%%%%%%%%%%%%%%%%%%%%%%%%%%%%%%%%%%%%%
\section{6. Zukunftsperspektiven: Digital vs. Analog}
%%%%%%%%%%%%%%%%%%%%%%%%%%%%%%%%%%%%%%%%%%%%%%%%%%%%%%%%%%%%%%%

\begin{frame}{Vom Analogen zum Digitalen}
  Moderne digitale Potentiometer (\emph{Digipots}) ersetzen den mechanischen Schleifer durch eine elektronische Widerstands-Schaltmatrix.
  \begin{itemize}
    \item \textbf{Vorteile:} Keine mechanische Abnutzung, kein CRV, präzise serielle Ansteuerung (I²C/SPI), geringe Größe.
    \item \textbf{Nachteile:} Begrenzte Auflösung (typ. 256–1024 Stufen), eingeschränkte Spannungs- und Leistungsbereiche.
  \end{itemize}
  \vspace{0.3cm}

\begin{center}
\scriptsize
\href{https://drive.google.com/file/d/1juGJN_5YOAVuYQdDns6diLQNnVhyoGma/view?usp=sharing}{\textbf{▶ Video anschauen}}
\end{center}

\end{frame}


% ---------------- Folie: Digitale vs. Analoge Potentiometer ----------------
\begin{frame}{Vergleich: Digital- vs. Analog-Potentiometer}
\centering

\begin{columns}[c]

    % --- Spalte 1: Analog ---
    \begin{column}{0.48\textwidth}
        \begin{figure}
            \centering
            \includegraphics[width=\textwidth, height=4.5cm, keepaspectratio]{digimech.jpg}
            \caption{Größenvergleich: Mechanisches Potentiometer (links) vs. Digitalpoti-IC (rechts).}
        \end{figure}
    \end{column}

    % --- Spalte 2: Digital ---
    \begin{column}{0.48\textwidth}
        \begin{figure}
            \centering
            \includegraphics[width=\textwidth, height=4.5cm, keepaspectratio]{digi.jpg}
            \caption{Digitaler Potentiometer (Digipot IC).}
        \end{figure}
    \end{column}

\end{columns}

\end{frame}

% ---------------- Folie: Kreativer Hinweis ----------------
\begin{frame}{Hinweis: Analoge vs. Digitale Potentiometer}
\centering
\begin{tikzpicture}
\node[
    draw=blue!50,             % Rahmenfarbe
    fill=blue!10,             % Hintergrundfarbe
    rounded corners=6mm,      % abgerundete Ecken
    drop shadow,              % Schatten für Tiefe
    text width=0.8\textwidth, % Breite des Textblocks
    align=center,             % zentrierter Text
    font=\footnotesize\itshape % Kursiv, kleinere Schrift
] 
{
\textbf{„Der Wandel ist die Umkehr der Kausalität.“} \\[1mm]
Das mechanische Poti ist ein \textit{passiver Transducer}, der eine Aktion erfasst.\\
Das digitale System ist ein \textit{aktives Stellglied}, das eine Information ausführt.
};
\end{tikzpicture}
\vspace{0.3cm}

\begin{center}
\scriptsize
\textbf{Von Hand zu Mikrochip:} \\
1841–1872: Mechanisches Poti – simple, elegant, handgesteuert. \\
1980er–1990er: Digitales Poti – winziger Chip, Transistoren orchestrieren Widerstände, präzise und zuverlässig. \\
Über 100 Jahre, und der Mensch verwandelte einfache Drehbewegungen in kontrollierbare digitale Intelligenz…
\end{center}

\end{frame}







\begin{frame}{Linearität im Vergleich: Mechanisch vs. Digital}
\textbf{Ziel:} Vergleich der Ausgangsspannung $U_{out}$ vs. Schleiferstellung / digitaler Schritt.

\begin{tikzpicture}[scale=1.0]
    % Achsen
    \draw[->, thick] (0,0) -- (6,0) node[right]{Schleiferposition / Schritt};
    \draw[->, thick] (0,0) -- (0,4.2) node[above]{$U_{out}$};

    % Mechanisch (belastet)
    \draw[red, thick, smooth] plot coordinates {(0,0) (1,0.68) (2,1.42) (3,2.05) (4,2.75) (5,3.48)};
    \node[red, anchor=west] at (3.5, 2.5) {Mechanisch (belastet)};

    % Digital (ideal)
    \draw[orange, thick, dashed] (0,0) -- (5,4.0);
    \node[orange, anchor=west] at (3.5, 3.8) {Digital (ideal linear)};
\end{tikzpicture}

\begin{itemize}
    \item Mechanisch: Kennlinie wird durch Last nichtlinear; Rauschen überlagert das Signal.
    \item Digital: Perfekt lineare Stufen (diskret), verschleiß- und rauschfrei.
\end{itemize}
\end{frame}

\begin{frame}{Rauschen & Belastungseffekt im Experiment}
\textbf{Setup:} Schleifer in mittlerer Position, konstanter Strom.

\begin{columns}[c]
    \begin{column}{0.45\textwidth}
        \begin{itemize}
            \item Drahtwicklung: CRV sichtbar, Schwankungen von \SIrange{50}{100}{\milli\volt}.
            \item Leitplastik: CRV deutlich geringer, ca. \SIrange{20}{50}{\milli\volt}.
            \item Digital: Praktisch kein Rauschen, stabile Ausgangsspannung.
        \end{itemize}
    \end{column}
    \begin{column}{0.55\textwidth}
        \begin{tikzpicture}[scale=0.9, every node/.style={scale=0.9}]
            % Achsen
            \draw[->, thick] (0,0) -- (6.5,0) node[below]{Zeit (s)};
            \draw[->, thick] (0,0) -- (0,2.5) node[left]{Spannung (mV)};
            % Grid
            \draw[gray!30, thin, step=0.5] (0,0) grid (6,2.2);
            % Signale (ohne 'rand' für maximale Kompatibilität)
            \draw[red, thick, domain=0:6, samples=200, smooth] plot (\x, {1.8+0.2*sin(15*\x r)});
            \draw[green, thick, domain=0:6, samples=200, smooth] plot (\x, {1.2+0.1*sin(12*\x r)});
            \draw[orange, thick] (0,0.5) -- (6,0.5);
            % Legende
            \node[red, anchor=west] at (1.5, 2.1) {Drahtwicklung};
            \node[green, anchor=west] at (1.5, 1.5) {Leitplastik};
            \node[orange, anchor=west] at (1.5, 0.8) {Digital};
        \end{tikzpicture}
    \end{column}
\end{columns}
\end{frame}

%%%%%%%%%%%%%%%%%%%%%%%%%%%%%%%%%%%%%%%%%%%%%%%%%%%%%%%%%%%%%%%
\section{7. Zusammenfassung}
%%%%%%%%%%%%%%%%%%%%%%%%%%%%%%%%%%%%%%%%%%%%%%%%%%%%%%%%%%%%%%%

\begin{frame}{Kernaussagen}
  \begin{enumerate}
    \item Potentiometer sind vielseitige, robuste elektromechanische Transducer.
    \item Die Realität weicht vom idealen Spannungsteiler ab (Belastungseffekt, CRV, parasitäre Effekte).
    \item Die Wahl des Widerstandsmaterials (Draht, Leitplastik) bestimmt Präzision, Lebensdauer und Rauschverhalten.
    \item Industrielle Anwendungen reichen von einfacher Steuerung bis zu hochzuverlässiger Sensorik in der Raumfahrt.
    \item Digitale Potentiometer bieten Verschleißfreiheit und Präzision, können mechanische Varianten aber nicht in allen Anwendungsfällen ersetzen.
  \end{enumerate}
\end{frame}




\begin{frame}{Danksagung}
    \vspace{0.5cm}
    \begin{center}
        \textbf{\Large Vielen Dank für Ihre Aufmerksamkeit!}
    \end{center}

    \vspace{0.5cm}
    \textbf{Besonderer Dank an:}
    \begin{itemize}
        \item Prof. Dr.-Ing. Björn Keune für die Betreuung des Seminars
        \item Die Technische Hochschule Georg Agricola (THGA) für die Unterstützung
        \item Alle KI-Tools (Google LLM Notebook, Gemini 2.5 Pro, Veo3, ElevenLab, Canva) für die Erstellung der Abbildungen und Videos
        \item Kommilitonen für wertvolles Feedback während der Vorbereitung
    \end{itemize}

    \vspace{0.5cm}
    \begin{center}
        \textit{Fragen oder Feedback sind herzlich willkommen.}
    \end{center}
\end{frame}



%%%%%%%%%%%%%%%%%%%%%%%%%%%%%%%%%%%%%%%%%%%%%%%%%%%%%%%%%%%%%%%
%
% HIER ENDET DER NEUE ABSCHNITT
%
% 
%
%%%%%%%%%%%%%%%%%%%%%%%%%%%%%%%%%%%%%%%%%%%%%%%%%%%%%%%%%%%%%%%
%%%%%%%%%%%%%%%%%%%%%%%%%%%%%%%%%%%%%%%%%%%%%%%%%%%%%%%%%%%%%%%
\section{8. Quellen}
%%%%%%%%%%%%%%%%%%%%%%%%%%%%%%%%%%%%%%%%%%%%%%%%%%%%%%%%%%%%%%%
\begin{frame}[allowframebreaks]{Quellen}
    \begin{thebibliography}{99}
        \bibitem{THGA_Vorlesung}
P. Shirafkan: 
\textit{Vorlesung: Bauelemente und Schaltungstechnik – Einführung und Widerstände.} 
Technische Hochschule Georg Agricola (THGA), internes Skript, 2024.
        \bibitem{WikiDigitalPot}
Wikipedia:
\textit{Digital Potentiometer.}
Abgerufen am 09.10.2025 von \url{https://en.wikipedia.org/wiki/Digital_potentiometer}
        \bibitem{SeminarLatex}
        Talea, Iliass:
        \textit{Seminar: Das Potentiometer.}
        LaTeX-Projekt, Technische Hochschule Georg Agricola, 2025.
        Verfügbar auf GitHub: \url{https://github.com/iliasstalea/Potentiometer}
    \end{thebibliography}
\end{frame}

%%%%%%%%%%%%%%%%%%%%%%%%%%%%%%%%%%%%%%%%%%%%%%%%%%%%%%%%%%%%%%%
\section{9. Bildnachweis} 
%%%%%%%%%%%%%%%%%%%%%%%%%%%%%%%%%%%%%%%%%%%%%%%%%%%%%%%%%%%%%%%
% ---------------- Folie 1: Bildnachweis Bild 1 ----------------
\begin{frame}{Bildnachweis – Positionssensoren}
    \textbf{Bild 1 Beispiel – Positionssensoren:}
    \begin{block}{Prompt:}
        /imagine a high-resolution, ultra-realistic macro photograph. Three different position sensors are lying on a dark, scratched, anti-static mat with visible dust specks.
        1.  On the left: A classic analog potentiometer, disassembled, showing the internal resistive track and a metal wiper with faint smudges.
        2.  In the middle: A tiny, black 8-pin digital potentiometer (Digipot) IC chip.
        3.  On the right: A contactless Hall-effect sensor, shown as two parts: a small silver neodymium magnet and the corresponding 3-pin Hall-sensor IC.
        The lighting is a single, harsh workshop lamp from above, creating hard shadows and highlighting textures like fingerprints and metal grain. This must look like a real, unedited photo, not a 3D render. Absolutely no text, labels, or writing.
    \end{block}

    \vspace{0.2cm}
    \textbf{Erstellt mit:} Google LLM Notebook, Gemini 2.5 Pro
\end{frame}

% ---------------- Folie 2: Bildnachweis Bild 2 ----------------
\begin{frame}{Bildnachweis – Potentiometer-Gruppe}
    \textbf{Bild 2 Beispiel – Potentiometer-Gruppe:}
    \begin{block}{Prompt:}
        /imagine a high-resolution, ultra-realistic macro photograph of three real-world electronic components resting on a dark, slightly scratched metal workbench. The group includes:
        1.  A standard rotary potentiometer with a metal shaft, showing tiny scratches and faint smudges or fingerprints on its casing.
        2.  A linear slide potentiometer.
        3.  A small blue 3-pin trimmer potentiometer (trim pot).
        A few specks of dust are visible on the components. The lighting is a single, bright overhead lamp, creating sharp, realistic shadows. The image must look like a real photo, not a 3D render. Absolutely no text, labels, or writing.
    \end{block}

    \vspace{0.2cm}
    \textbf{Erstellt mit:} Google LLM Notebook, Gemini 2.5 Pro
\end{frame}

% ---------------- Folie 3: Video ----------------


\begin{frame}[fragile]{Video-Nachweis – „Reise durch die Welt der Potentiometer“}
\textbf{Video-Prompt JSON :}

\vspace{0.3cm}
\begin{lstlisting}[language=json]
{
  "prompt": "Ultra-realistic 8s video: 
  1. Hand turns volume knob, camera shows analog Poti and VU meter. 
  2. Macro wire-wound Poti, copper coil, pins, base. 
  3. 360° rotation of conductive-plastic Poti, transparent knob, stacked discs. 
  4. Close-up guitar volume/tone knob. 
  5. Airplane exterior to cockpit, fly-through, potentiometers in systems. 
  6. Cockpit close-up: instruments, pilots. 
  7. Conclusion: 'Vielen Dank' message ... 
  Tools: Veo3, ElevenLab, Canva",
  "duration": 8
}
\end{lstlisting}

\vspace{0.2cm}
\scriptsize
\textit{Hinweis: Dies ist eine stark verkürzte Version des Prompts. Die vollständige Version ist deutlich länger und würde viele Seiten füllen.}
\end{frame}

\begin{frame}[fragile]{Video Prompt JSON – Digital Potentiometer Demo}
\textbf{Video-Prompt JSON :}

\vspace{0.3cm}
\begin{lstlisting}[language=json]
{
  "prompt": "Ultra-realistic macro 8s video: working 8-pin Digipot circuit on cluttered workbench.
  Round 1: Arduino Nano sends commands, red LED dim.
  Round 2: LED steps up to medium brightness.
  Round 3: LED steps to full brightness, discrete steps.
  Round 4: LED steps down to medium then dim, repeating.
  Realistic lighting, blurred surroundings, dust, metallic reflections, no text or labels.",
  Tools: Veo3, 
  First Foto Frames :  Leonardo Ai
  "duration": 8
 "
}
\end{lstlisting}
\end{frame}












\end{frame}







\end{document}