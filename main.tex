%%%%%%%%%%%%%%%%%%%%%%%%%%%%%%%%%%%%%%%%%%%%%%%%%%%%%%%%%%%%%%%
%
% Seminar: Potentiometer 
% Iliass Talea
%
%%%%%%%%%%%%%%%%%%%%%%%%%%%%%%%%%%%%%%%%%%%%%%%%%%%%%%%%%%%%%%%
\documentclass[aspectratio=169]{beamer}

% --- THEMA UND FARBEN ---
\usetheme{Madrid}
\usecolortheme{beaver}

% --- PAKETE ---
\usepackage[utf8]{inputenc}
\usepackage[T1]{fontenc}
\usepackage[german]{babel}
\usepackage{amsmath}
\usepackage{graphicx}
\usepackage{booktabs}
\usepackage{siunitx} % Für korrekte Darstellung von Einheiten
\usepackage{tikz}
\usetikzlibrary{circuits.ee.IEC, positioning, arrows.meta}
\usepackage{microtype} % Verbessert die Typografie
\usepackage{hyperref} % Für klickbare Links im Quellenverzeichnis
\hypersetup{colorlinks=true, urlcolor=blue, linkcolor=black}
% --- ADD THIS LINE TO TURN ON CAPTION NUMBERING ---
\setbeamertemplate{caption}[numbered]

% --- TITELINFORMATIONEN ---
\title[Potentiometer]{Seminar: Das Potentiometer}
\subtitle{Analyse eines elektromechanischen Transducers}
\author{Iliass Talea}
\institute{THGA}
\date{\today}

% --- FUSSZEILE ANPASSEN (KORRIGIERTE VERSION) ---

\setbeamertemplate{footline}
{
  \leavevmode%
  \hbox{%
  \begin{beamercolorbox}[wd=.4\paperwidth,ht=2.25ex,dp=1ex,left]{author in head/foot}%
    \hspace{2ex}\usebeamerfont{author in head/foot}\insertshortauthor
  \end{beamercolorbox}%
  \begin{beamercolorbox}[wd=.2\paperwidth,ht=2.25ex,dp=1ex,center]{title in head/foot}%
    \usebeamerfont{title in head/foot}\insertshorttitle
  \end{beamercolorbox}%
  \begin{beamercolorbox}[wd=.4\paperwidth,ht=2.25ex,dp=1ex,right]{date in head/foot}%
    \usebeamerfont{date in head/foot}\insertshortdate{}\hspace*{2em}%
    \insertframenumber{} / \inserttotalframenumber\hspace*{1ex}%
    % Logo ist jetzt HIER integriert, um Layout-Fehler zu vermeiden
    \includegraphics[height=1.8ex]{THGA-Logo.png}\hspace*{2ex}
  \end{beamercolorbox}}%
  \vskip0pt%
}


\begin{document}

% --- TITELFOLIE (angepasst, professionell im Seminar-Design) ---
{
  \setbeamercolor{background canvas}{bg=white} % weißer Hintergrund
  \begin{frame}[plain]

    % Hintergrundbalken
    \begin{tikzpicture}[remember picture,overlay]
      \fill[red!70!black] (current page.north west) rectangle ([yshift=-3cm]current page.north east);
    \end{tikzpicture}

    % Titel oben im Balken (weiß)
    \vspace*{0.3cm}
    {\color{white}\Huge \textbf{Seminar: Das Potentiometer}} \\[0.3cm]
    {\color{white}\large Analyse eines elektromechanischen Transducers}

    \vfill

    % Autor / Uni unten links
    \begin{flushleft}
      \textbf{Iliass Talea} \\ 
      Technische Hochschule Georg Agricola (THGA) \\ 
      Seminarleiter: Prof. Dr.-Ing. Björn Keune \\ 
      \today
    \end{flushleft}

    % Logo unten rechts
    \begin{flushright}
      \includegraphics[height=1.5cm]{THGA-Logo.png}
    \end{flushright}

  \end{frame}
}

% --- INHALTSVERZEICHNIS ---
\begin{frame}{Agenda}
  \tableofcontents
\end{frame}

%%%%%%%%%%%%%%%%%%%%%%%%%%%%%%%%%%%%%%%%%%%%%%%%%%%%%%%%%%%%%%%
\section{1. Einleitung und Klassifizierung}
%%%%%%%%%%%%%%%%%%%%%%%%%%%%%%%%%%%%%%%%%%%%%%%%%%%%%%%%%%%%%%%

\begin{frame}{Das Potentiometer als elektromechanischer Transducer}
    
    
    \begin{columns}[c]
        
        \begin{column}{0.45\textwidth}
            \begin{figure}
                
                \includegraphics[width=\textwidth, height=5cm, keepaspectratio]{aufbaupoti.jpg}
                \caption{Prinzipieller Aufbau eines Drehpotis.}
            \end{figure}
        \end{column}

        \begin{column}{0.55\textwidth}
            \begin{block}{ Definition}
                Ein Potentiometer ist ein passiver, elektromechanischer Wandler, der eine mechanische Position in eine proportionale Widerstandsänderung umwandelt.
            \end{block}
            
            \medskip 
            
            \textbf{Kernkomponenten:}
            \begin{itemize}
                \item Stationäres Widerstandselement
                \item Beweglicher Schleiferkontakt
                \item Drei elektrische Anschlüsse
            \end{itemize}
        \end{column}

    \end{columns}
\end{frame}

%%%%%%%%%%%%%%%%%%%%%%%%%%%%%%%%%%%%%%%%%%%%%%%%%%%%%%%%%%%%%%%
\section{2. Physikalische Realisierung}
%%%%%%%%%%%%%%%%%%%%%%%%%%%%%%%%%%%%%%%%%%%%%%%%%%%%%%%%%%%%%%%

\begin{frame}{Mechanische Potentiometer im Überblick}
    \begin{columns}[T]
        \begin{column}{0.55\textwidth}
            \begin{block}{Drahtwickel-Potentiometer}
                \begin{itemize}
                    \item Widerstand aus fein gewickeltem Draht
                    \item Hohe Linearität und Belastbarkeit
                    \item Endliche Auflösung durch Drahtwicklungen
                    \item Ideal für präzise Messgeräte und industrielle Sensorik
                \end{itemize}
            \end{block}

            \begin{block}{Leitplastik-Potentiometer}
                \begin{itemize}
                    \item Widerstandsbahn aus leitfähigem Kunststoff
                    \item Kostengünstig und quasi unendliche Auflösung
                    \item Geringeres Kontaktrauschen
                    \item Typisch in Audio- und Konsumelektronik
                \end{itemize}
            \end{block}
        \end{column}
        \begin{column}{0.45\textwidth}
            \begin{figure}
                % <<< THE FIX IS HERE
                % We limit the height to 6cm and keep the aspect ratio.
                \includegraphics[width=\textwidth, height=5.5cm, keepaspectratio]{plastik.png}
                
                \caption{\textcolor{black}{Drahtwickel- (links) vs. Leitplastik-Potentiometer (rechts).}}
            \end{figure}
        \end{column}
    \end{columns}
\end{frame}

%%%%%%%%%%%%%%%%%%%%%%%%%%%%%%%%%%%%%%%%%%%%%%%%%%%%%%%%%%%%%%%
\section{3. Analyse der elektrischen Eigenschaften}
%%%%%%%%%%%%%%%%%%%%%%%%%%%%%%%%%%%%%%%%%%%%%%%%%%%%%%%%%%%%%%%

\begin{frame}{Der ideale Spannungsteiler vs. Realität}
    Im Idealfall (unbelasteter Ausgang) ist die Ausgangsspannung eines linearen Potentiometers exakt proportional zur Position des Schleifers:
    \[ U_{out}(a) = U_{in} \cdot a \quad \text{mit} \quad a \in [0, 1] \]
    wobei $a$ die normalisierte Schleiferposition ist.

    \pause

    \begin{beamercolorbox}[sep=1ex,center,wd=\textwidth]{block title}
        Die Realität ist jedoch komplexer!
    \end{beamercolorbox}
\end{frame}

\begin{frame}{Analyse der Realität: Störfaktoren}
    \begin{alertblock}{In der Praxis wird die ideale Übertragungsfunktion durch mehrere Faktoren gestört:}
    \begin{itemize}
        \item Der \textbf{Belastungseffekt} durch den Lastwiderstand
        \item Die \textbf{endliche Auflösung} (bei Drahtwickel-Potis)
        \item Das \textbf{Kontaktwiderstandsrauschen (CRV)}
        \item \textbf{Parasitäre Effekte} bei hohen Frequenzen
    \end{itemize}
    \end{alertblock}
\end{frame}

\begin{frame}{Kritischer Faktor: Der Belastungseffekt}
    \textbf{Problem:} Der an den Schleifer angeschlossene Lastwiderstand $R_L$ bildet eine Parallelschaltung mit dem Teilwiderstand $R_2$.

    \begin{columns}[c]
        \begin{column}{0.55\textwidth}
            Dies führt zu einer \textbf{nichtlinearen} Übertragungsfunktion.
            \begin{block}{Spannungsteilerformel (belastet)}
              \[ U_{out} = U_{in} \cdot \frac{R_2 || R_L}{R_1 + (R_2 || R_L)} \]
            \end{block}
            \vspace{0.2cm}
            Mit $R_2||R_L = \tfrac{R_2 R_L}{R_2+R_L}$. Der Fehler ist maximal, wenn $R_L$ in der Größenordnung des Potentiometerwiderstands liegt.
        \end{column}
        \begin{column}{0.45\textwidth}
            \begin{tikzpicture}[circuit ee IEC, thick, scale=0.9]
              \draw (0,0) to[resistor={info=$R_1$}] (0,-2)
                          to[resistor={info=$R_2$}] (0,-4) node[ground]{};
              \draw (0,-2) -- (2,-2)
                          to[resistor={info=$R_L$}] (2,-4) node[ground]{};
              \node at (-0.5,0) {$U_{in}$};
              \node at (2.5,-2) {$U_{out}$};
            \end{tikzpicture}
        \end{column}
    \end{columns}
\end{frame}

%%%%%%%%%%%%%%%%%%%%%%%%%%%%%%%%%%%%%%%%%%%%%%%%%%%%%%%%%%%%%%%
\section{4. Erweiterte Betrachtungen}
%%%%%%%%%%%%%%%%%%%%%%%%%%%%%%%%%%%%%%%%%%%%%%%%%%%%%%%%%%%%%%%

\begin{frame}{Kontaktwiderstandsrauschen (CRV)}
    \textbf{Definition:} Fluktuationen im Übergangswiderstand zwischen Schleifer und Widerstandsbahn, die eine Rauschspannung erzeugen. \\
    
    \textbf{Mathematisches Modell:}
    \[ U_{noise}(t) \approx k \cdot I(t) \cdot \Delta R_c(t) \]
    mit $\Delta R_c$ = zeitabhängige Schwankung des Kontaktwiderstands.

    \begin{exampleblock}{Industrielle Relevanz}
        In Audioverstärkern führt CRV zu hörbarem „Kratzen“ beim Verstellen der Lautstärke. Hochwertige Leitplastik-Potis minimieren diesen Effekt.
    \end{exampleblock}
\end{frame}

\begin{frame}{Parasitäre Induktivität und Kapazität}
    Drahtgewickelte Potentiometer zeigen parasitäre Induktivitäten $L_p$, während großflächige Bahnen parasitäre Kapazitäten $C_p$ erzeugen.

    \begin{block}{Ersatzschaltbild bei hohen Frequenzen}
    \[ Z_{eff}(\omega) = R + j\omega L_p + \frac{1}{j\omega C_p} \]
    \end{block}

    \begin{itemize}
        \item Bedeutend bei HF-Anwendungen (z. B. Radar, Kommunikation).
        \item Lösung: Einsatz von \textbf{speziellen HF-Potis} oder \textbf{digitale Alternativen}.
    \end{itemize}
\end{frame}

%%%%%%%%%%%%%%%%%%%%%%%%%%%%%%%%%%%%%%%%%%%%%%%%%%%%%%%%%%%%%%%
\section{5. Industrielle Anwendungen}
%%%%%%%%%%%%%%%%%%%%%%%%%%%%%%%%%%%%%%%%%%%%%%%%%%%%%%%%%%%%%%%

\begin{frame}{Anwendungen: Luft- und Raumfahrt}
    \begin{columns}[c] 

      % --- COLUMN 1: TEXT ---
      \begin{column}{0.5\textwidth}
        \begin{itemize}
          \item Positionssensoren in \textbf{Fly-by-Wire}-Systemen (z.B. Joystick-Abfrage).
          \item Steuerung der Ausrichtung von \textbf{Satellitenantennen}.
          \item Hohe Robustheit gegen Strahlungseinflüsse (besonders drahtgewickelte Varianten).
        \end{itemize}
      \end{column}

      % --- COLUMN 2: IMAGE ---
      \begin{column}{0.5\textwidth}
        \begin{figure}
            \centering
            \includegraphics[width=150]{potisaerospace.jpg}
            
            
            
            \vspace{0cm} 
            
           
            \caption{Hochpräzisions-Potentiometer für die Luftfahrtsteuerung.}
        \end{figure}
      \end{column}

    \end{columns}
\end{frame}

\begin{frame}{Anwendungen: Robotik und Medizintechnik}
    \begin{block}{Robotik}
      Gelenkwinkelmessung in Industrierobotern: Eine robuste und kostengünstige Methode zur Positionsrückführung in rauen Umgebungen.
    \end{block}
    \begin{block}{Medizintechnik}
      Exakte Positionierung in Infusionspumpen oder die Einstellung von Parametern an Dialysegeräten.
    \end{block}
    \vspace{0.3cm}
    Alternative: Kontaktlose \textbf{Hall-Sensoren} gewinnen an Bedeutung, sind jedoch oft teurer und temperaturabhängiger.
\end{frame}

%%%%%%%%%%%%%%%%%%%%%%%%%%%%%%%%%%%%%%%%%%%%%%%%%%%%%%%%%%%%%%%
\section{6. Zukunftsperspektiven: Digital vs. Analog}
%%%%%%%%%%%%%%%%%%%%%%%%%%%%%%%%%%%%%%%%%%%%%%%%%%%%%%%%%%%%%%%

\begin{frame}{Vom Analogen zum Digitalen}
  Moderne digitale Potentiometer (\emph{Digipots}) ersetzen den mechanischen Schleifer durch eine elektronische Widerstands-Schaltmatrix.
  \begin{itemize}
    \item \textbf{Vorteile:} Keine mechanische Abnutzung, kein CRV, präzise serielle Ansteuerung (I²C/SPI), geringe Größe.
    \item \textbf{Nachteile:} Begrenzte Auflösung (typ. 256–1024 Stufen), eingeschränkte Spannungs- und Leistungsbereiche.
  \end{itemize}
\end{frame}

\begin{frame}{Linearität im Vergleich: Mechanisch vs. Digital}
\textbf{Ziel:} Vergleich der Ausgangsspannung $U_{out}$ vs. Schleiferstellung / digitaler Schritt.

\begin{tikzpicture}[scale=1.0]
    % Achsen
    \draw[->, thick] (0,0) -- (6,0) node[right]{Schleiferposition / Schritt};
    \draw[->, thick] (0,0) -- (0,4.2) node[above]{$U_{out}$};

    % Mechanisch (belastet)
    \draw[red, thick, smooth] plot coordinates {(0,0) (1,0.68) (2,1.42) (3,2.05) (4,2.75) (5,3.48)};
    \node[red, anchor=west] at (3.5, 2.5) {Mechanisch (belastet)};

    % Digital (ideal)
    \draw[orange, thick, dashed] (0,0) -- (5,4.0);
    \node[orange, anchor=west] at (3.5, 3.8) {Digital (ideal linear)};
\end{tikzpicture}

\begin{itemize}
    \item Mechanisch: Kennlinie wird durch Last nichtlinear; Rauschen überlagert das Signal.
    \item Digital: Perfekt lineare Stufen (diskret), verschleiß- und rauschfrei.
\end{itemize}
\end{frame}

\begin{frame}{Rauschen & Belastungseffekt im Experiment}
\textbf{Setup:} Schleifer in mittlerer Position, konstanter Strom.

\begin{columns}[c]
    \begin{column}{0.45\textwidth}
        \begin{itemize}
            \item Drahtwicklung: CRV sichtbar, Schwankungen von \SIrange{50}{100}{\milli\volt}.
            \item Leitplastik: CRV deutlich geringer, ca. \SIrange{20}{50}{\milli\volt}.
            \item Digital: Praktisch kein Rauschen, stabile Ausgangsspannung.
        \end{itemize}
    \end{column}
    \begin{column}{0.55\textwidth}
        \begin{tikzpicture}[scale=0.9, every node/.style={scale=0.9}]
            % Achsen
            \draw[->, thick] (0,0) -- (6.5,0) node[below]{Zeit (s)};
            \draw[->, thick] (0,0) -- (0,2.5) node[left]{Spannung (mV)};
            % Grid
            \draw[gray!30, thin, step=0.5] (0,0) grid (6,2.2);
            % Signale (ohne 'rand' für maximale Kompatibilität)
            \draw[red, thick, domain=0:6, samples=200, smooth] plot (\x, {1.8+0.2*sin(15*\x r)});
            \draw[green, thick, domain=0:6, samples=200, smooth] plot (\x, {1.2+0.1*sin(12*\x r)});
            \draw[orange, thick] (0,0.5) -- (6,0.5);
            % Legende
            \node[red, anchor=west] at (1.5, 2.1) {Drahtwicklung};
            \node[green, anchor=west] at (1.5, 1.5) {Leitplastik};
            \node[orange, anchor=west] at (1.5, 0.8) {Digital};
        \end{tikzpicture}
    \end{column}
\end{columns}
\end{frame}

%%%%%%%%%%%%%%%%%%%%%%%%%%%%%%%%%%%%%%%%%%%%%%%%%%%%%%%%%%%%%%%
\section{7. Zusammenfassung}
%%%%%%%%%%%%%%%%%%%%%%%%%%%%%%%%%%%%%%%%%%%%%%%%%%%%%%%%%%%%%%%

\begin{frame}{Kernaussagen}
  \begin{enumerate}
    \item Potentiometer sind vielseitige, robuste elektromechanische Transducer.
    \item Die Realität weicht vom idealen Spannungsteiler ab (Belastungseffekt, CRV, parasitäre Effekte).
    \item Die Wahl des Widerstandsmaterials (Draht, Leitplastik) bestimmt Präzision, Lebensdauer und Rauschverhalten.
    \item Industrielle Anwendungen reichen von einfacher Steuerung bis zu hochzuverlässiger Sensorik in der Raumfahrt.
    \item Digitale Potentiometer bieten Verschleißfreiheit und Präzision, können mechanische Varianten aber nicht in allen Anwendungsfällen ersetzen.
  \end{enumerate}
\end{frame}


%%%%%%%%%%%%%%%%%%%%%%%%%%%%%%%%%%%%%%%%%%%%%%%%%%%%%%%%%%%%%%%
%
% HIER BEGINNT DER NEUE ABSCHNITT 
%
%%%%%%%%%%%%%%%%%%%%%%%%%%%%%%%%%%%%%%%%%%%%%%%%%%%%%%%%%%%%%%%
\section{10. Diskussion} 

\begin{frame}{Diskussionsfrage 1: Belastung}
    \begin{block}{Szenario}
        \begin{itemize}
            \item Poti-Widerstand = \SI{10}{\kilo\ohm}
            \item Last-Widerstand = \SI{10}{\kilo\ohm}
        \end{itemize}
    \end{block}

    \begin{alertblock}{Frage}
        Was passiert hier mit der Ausgangsspannung?
    \end{alertblock}
    
    \pause % <-- Klick
    \vfill
    \begin{exampleblock}{Antwort}
        \begin{itemize}
            \item Der \textbf{Belastungseffekt} tritt auf.
            \item Die Spannung wird \textbf{nichtlinear} (falsch).
        \end{itemize}
    \end{exampleblock}
\end{frame}

% --- ZWEITE FRAGE ---

\begin{frame}{Diskussionsfrage 2: Material}
    \begin{alertblock}{Frage}
        Welches Poti für ein Audio-Gerät (damit es nicht "kratzt"): Drahtwickel oder Leitplastik?
    \end{alertblock}
    
    \pause % <-- Klick
    \vfill
    \begin{exampleblock}{Antwort}
        \begin{itemize}
            \item \textbf{Leitplastik}.
            \item Grund: Es hat fast kein \textbf{CRV} (Rauschen).
        \end{itemize}
    \end{exampleblock}
\end{frame}

% --- DRITTE FRAGE ---

\begin{frame}{Diskussionsfrage 3: Digital}
    \begin{alertblock}{Frage}
        Was ist der größte Vorteil eines \textbf{Digital-Potis} (Digipot) im Vergleich zu einem mechanischen?
    \end{alertblock}
    
    \pause % <-- Klick
    \vfill
    \begin{exampleblock}{Antwort}
        \begin{itemize}
            \item \textbf{Kein mechanischer Verschleiß.}
            \item (Und auch kein Rauschen / CRV).
        \end{itemize}
    \end{exampleblock}
\end{frame}

%%%%%%%%%%%%%%%%%%%%%%%%%%%%%%%%%%%%%%%%%%%%%%%%%%%%%%%%%%%%%%%
%
% HIER ENDET DER NEUE ABSCHNITT
%
% 
%
%%%%%%%%%%%%%%%%%%%%%%%%%%%%%%%%%%%%%%%%%%%%%%%%%%%%%%%%%%%%%%%
%%%%%%%%%%%%%%%%%%%%%%%%%%%%%%%%%%%%%%%%%%%%%%%%%%%%%%%%%%%%%%%
\section{8. Quellen}
%%%%%%%%%%%%%%%%%%%%%%%%%%%%%%%%%%%%%%%%%%%%%%%%%%%%%%%%%%%%%%%
\begin{frame}[allowframebreaks]{Quellen}
    \begin{thebibliography}{99}
        \bibitem{THGA_Vorlesung}
P. Shirafkan: 
\textit{Vorlesung: Bauelemente und Schaltungstechnik – Einführung und Widerstände.} 
Technische Hochschule Georg Agricola (THGA), internes Skript, 2024.
        \bibitem{WikiDigitalPot}
Wikipedia:
\textit{Digital Potentiometer.}
Abgerufen am 09.10.2025 von \url{https://en.wikipedia.org/wiki/Digital_potentiometer}
        \bibitem{SeminarLatex}
        Talea, Iliass:
        \textit{Seminar: Das Potentiometer.}
        LaTeX-Projekt, Technische Hochschule Georg Agricola, 2025.
        Verfügbar auf GitHub: \url{https://github.com/iliasstalea/Potentiometer}
    \end{thebibliography}
\end{frame}

%%%%%%%%%%%%%%%%%%%%%%%%%%%%%%%%%%%%%%%%%%%%%%%%%%%%%%%%%%%%%%%
\section{9. Bildnachweis} 
%%%%%%%%%%%%%%%%%%%%%%%%%%%%%%%%%%%%%%%%%%%%%%%%%%%%%%%%%%%%%%%
\begin{frame}{Bildnachweis und Danksagung}
    \textbf{Erstellung der Abbildungen:}
    \begin{itemize}
        \item Alle Abbildungen in dieser Präsentation wurden mithilfe von KI-Werkzeugen (Google LLM Notebook und Gemini 2.5 Pro) durch den Autor erstellt.
    \end{itemize}
    \begin{block}{Beispielprompt (verkürzt)}
        \textit{“Create a clear educational diagram comparing an analog rotary potentiometer and a digital potentiometer, showing the internal resistive track and wiper for the analog, and the resistor ladder network with switches for the digital. Use IEC symbols.”}
    \end{block}
\end{frame}

%%%%%%%%%%%%%%%%%%%%%%%%%%%%%%%%%%%%%%%%%%%%%%%%%%%%%%%%%%%%%%%
\begin{frame}[standout]
  Fragen?
\end{frame}









\end{document}