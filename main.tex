%%%%%%%%%%%%%%%%%%%%%%%%%%%%%%%%%%%%%%%%%%%%%%%%%%%%%%%%%%%%%%%
%
% Experten-Seminar: Potentiometer (Version 2.3 - ERWEITERT)
% Mit vertieften Formeln, Schaltplänen und industriellen Beispielen
%
%%%%%%%%%%%%%%%%%%%%%%%%%%%%%%%%%%%%%%%%%%%%%%%%%%%%%%%%%%%%%%%
\documentclass[aspectratio=169]{beamer}

% --- THEMA UND FARBEN ---
\usetheme{Madrid}
\usecolortheme{beaver}

% --- PAKETE ---
\usepackage[utf8]{inputenc}
\usepackage[T1]{fontenc}
\usepackage[german]{babel}
\usepackage{amsmath}
\usepackage{graphicx}
\usepackage{booktabs}
\usepackage{siunitx}
\usepackage{tikz}
\usetikzlibrary{circuits.ee.IEC, positioning, arrows.meta}
\usepackage{microtype}

% --- TITELINFORMATIONEN ---
\title[Potentiometer]{Seminar: Das Potentiometer}
\subtitle{Analyse eines elektromechanischen Transducers}
\author{Iliass Talea}
\institute{THGA}
\date{\today}

% --- FUSSZEILE ANPASSEN ---
\addtobeamertemplate{footline}{}{
  \leavevmode%
  \begin{beamercolorbox}[wd=0.1\paperwidth,ht=0.5cm,dp=0.1cm,right]{palette primary}%
    \includegraphics[height=0.5cm]{THGA-Logo.svg.png}% Achte auf PNG statt SVG
    \hspace*{0.2cm}%
  \end{beamercolorbox}%
}

\begin{document}

% --- TITELFOLIE ---
% --- TITELFOLIE (angepasst, professionell im Seminar-Design) ---
{
  \setbeamercolor{background canvas}{bg=white} % weißer Hintergrund
  \begin{frame}[plain]

    % Hintergrundbalken
    \begin{tikzpicture}[remember picture,overlay]
      \fill[red!70!black] (current page.north west) rectangle ([yshift=-3cm]current page.north east);
    \end{tikzpicture}

    % Titel oben im Balken (weiß)
    \vspace*{0.3cm}
    {\color{white}\Huge \textbf{Seminar: Das Potentiometer}} \\[0.3cm]
    {\color{white}\large Analyse eines elektromechanischen Transducers}

    \vfill

    % Autor / Uni unten links
    \begin{flushleft}
      \textbf{Iliass Talea} \\ 
      Technische Hochschule Georg Agricola (THGA) \\ 
      Seminarleiter: Prof. Dr.-Ing. Björn Keune \\ 
      \today
    \end{flushleft}

    % Logo unten rechts
    \begin{flushright}
      \includegraphics[height=1.5cm]{THGA-Logo.svg.png}
    \end{flushright}

  \end{frame}
}



% --- INHALTSVERZEICHNIS ---
\begin{frame}{Agenda}
  \scriptsize % noch kompakter
  \tableofcontents[hidesubsections]
\end{frame}



%%%%%%%%%%%%%%%%%%%%%%%%%%%%%%%%%%%%%%%%%%%%%%%%%%%%%%%%%%%%%%%
\section{1. Einleitung und Klassifizierung}
%%%%%%%%%%%%%%%%%%%%%%%%%%%%%%%%%%%%%%%%%%%%%%%%%%%%%%%%%%%%%%%

\begin{frame}{Das Potentiometer als elektromechanischer Transducer}
    \begin{block}{Formale Definition}
        Ein Potentiometer ist ein passiver, elektromechanischer Wandler (Transducer), der eine mechanische Positionsänderung (linear oder rotatorisch) in eine proportionale Änderung eines elektrischen Widerstandswertes umwandelt. Primär wird es als einstellbarer Spannungsteiler eingesetzt.
    \end{block}
    
    % \vspace{0.5cm}  % <<< Temporär auskommentieren
    
    \begin{columns}[c]
        \begin{column}{0.5\textwidth}
            \textbf{Kernkomponenten:}
            \begin{itemize}
                \item Stationäres Widerstandselement
                \item Beweglicher Schleiferkontakt
                \item Drei elektrische Anschlüsse
            \end{itemize}
        \end{column}
        \begin{column}{0.5\textwidth}
             % Test: kleines Dummy-Bild
             \includegraphics[width=\textwidth,height=4cm]{Potentiometer1.png}
\caption*{\tiny \textbf{Abb. 1:} Prinzipieller Aufbau eines Drehpotentiometers}
        \end{column}
    \end{columns}
\end{frame}


%%%%%%%%%%%%%%%%%%%%%%%%%%%%%%%%%%%%%%%%%%%%%%%%%%%%%%%%%%%%%%%
\section{2. Physikalische Realisierung}
%%%%%%%%%%%%%%%%%%%%%%%%%%%%%%%%%%%%%%%%%%%%%%%%%%%%%%%%%%%%%%%

% --- Slides über Materialien etc. (unverändert) ---

%%%%%%%%%%%%%%%%%%%%%%%%%%%%%%%%%%%%%%%%%%%%%%%%%%%%%%%%%%%%%%%
\section{3. Analyse der elektrischen Eigenschaften}
%%%%%%%%%%%%%%%%%%%%%%%%%%%%%%%%%%%%%%%%%%%%%%%%%%%%%%%%%%%%%%%

\begin{frame}{Der ideale Spannungsteiler vs. Realität}
    Im Idealfall ist die Ausgangsspannung eines linearen Potentiometers exakt proportional zur Position des Schleifers:
    $$ U_{out}(a) = U_{in} \cdot a \quad \text{mit} \quad a \in [0, 1] $$

    \pause

    \begin{beamercolorbox}[sep=1ex,center,wd=\textwidth]{block title}
        Die Realität ist jedoch komplexer!
    \end{beamercolorbox}
\end{frame}

\begin{frame}{Analyse der Realität: Störfaktoren}
    \begin{alertblock}{In der Praxis wird die ideale Übertragungsfunktion durch mehrere Faktoren gestört:}
    \begin{itemize}
        \item Der \textbf{Belastungseffekt (Loading Effect)}
        \item Die \textbf{endliche Auflösung} (bei Drahtwickel-Potis)
        \item Das \textbf{Kontaktwiderstandsrauschen (CRV)}
        \item \textbf{Parasitäre Effekte} bei hohen Frequenzen
    \end{itemize}
    \end{alertblock}
\end{frame}

\begin{frame}{Kritischer Faktor: Der Belastungseffekt (Loading Effect)}
\textbf{Problem:} Der an den Schleifer angeschlossene Lastwiderstand $R_L$ bildet eine Parallelschaltung mit dem Teilwiderstand $R_2$.

\begin{columns}[c]
    \begin{column}{0.55\textwidth}
        Dies führt zu einer \textbf{nichtlinearen} Übertragungsfunktion.
        \begin{block}{Spannungsteilerformel (belastet)}
          $$ U_{out} = U_{in} \cdot \frac{R_2 || R_L}{R_1 + (R_2 || R_L)} $$
        \end{block}
        \vspace{0.3cm}
        Mit $R_2||R_L = \tfrac{R_2 R_L}{R_2+R_L}$.
    \end{column}
    \begin{column}{0.45\textwidth}
        \begin{tikzpicture}[circuit ee IEC, thick, scale=0.9]
          \draw (0,0) to[resistor={info=$R_1$}] (0,-2)
                    to[resistor={info=$R_2$}] (0,-4) node[ground]{};
          \draw (0,-2) -- (2,-2)
                    to[resistor={info=$R_L$}] (2,-4) node[ground]{};
          \node at (-0.5,0) {$U_{in}$};
          \node at (2.5,-2) {$U_{out}$};
        \end{tikzpicture}
    \end{column}
\end{columns}
\end{frame}

%%%%%%%%%%%%%%%%%%%%%%%%%%%%%%%%%%%%%%%%%%%%%%%%%%%%%%%%%%%%%%%
\section{4. Erweiterte Betrachtungen}
%%%%%%%%%%%%%%%%%%%%%%%%%%%%%%%%%%%%%%%%%%%%%%%%%%%%%%%%%%%%%%%

\begin{frame}{Kontaktwiderstandsrauschen (CRV)}
    \textbf{Definition:} Fluktuationen im Übergangswiderstand zwischen Schleifer und Bahn. \\
    \vspace{0.3cm}
    \textbf{Mathematisches Modell:}
    $$ U_{noise}(t) \approx k \cdot I(t) \cdot \Delta R_c(t) $$
    mit $\Delta R_c$ = zeitabhängige Schwankung des Kontaktwiderstands.

    \vspace{0.5cm}
    \begin{exampleblock}{Industrielle Relevanz}
        In Audioverstärkern führt CRV zu hörbarem „Kratzen“. Hochwertige Leitplastik-Potis minimieren diesen Effekt.
    \end{exampleblock}
\end{frame}

\begin{frame}{Parasitäre Induktivität und Kapazität}
    Drahtgewickelte Potentiometer zeigen parasitäre Induktivitäten $L_p$, während großflächige Bahnen Kapazitäten $C_p$ erzeugen.

    \begin{block}{Ersatzschaltbild (vereinfachte Darstellung)}
    $$ Z_{eff}(\omega) = R + j\omega L_p + \frac{1}{j\omega C_p} $$
    \end{block}

    \begin{itemize}
        \item Bedeutend bei HF-Anwendungen (z. B. Radar, Kommunikation).
        \item Lösung: Einsatz von \textbf{leitplastischen Potis} oder \textbf{digitale Alternativen}.
    \end{itemize}
\end{frame}

\begin{frame}{Industrielle Anwendungen: Luft- und Raumfahrt}
    \begin{columns}
      \begin{column}{0.6\textwidth}
        \begin{itemize}
          \item Positionssensoren in \textbf{Fly-by-Wire}-Systemen.
          \item Steuerung von \textbf{Satellitenantennen}.
          \item Robust gegen Strahlungseinflüsse, wenn drahtgewickelt.
        \end{itemize}
      \end{column}
      \begin{column}{0.4\textwidth}
        \includegraphics[width=\textwidth]{Potentiometer2.png}
        \tiny{Potentiometer in der Luftfahrtsteuerung}
      \end{column}
    \end{columns}
\end{frame}

\begin{frame}{Industrielle Anwendungen: Robotik und Medizintechnik}
    \begin{block}{Robotik}
      Gelenkwinkelmessung in Industrierobotern: robuste Rückführung in rauen Umgebungen.
    \end{block}
    \begin{block}{Medizintechnik}
      Exakte Positionierung in Infusionspumpen oder chirurgischen Instrumenten.
    \end{block}
    \vspace{0.3cm}
    Alternative: \textbf{Hall-Sensoren} gewinnen an Bedeutung, jedoch oft teurer.
\end{frame}

%%%%%%%%%%%%%%%%%%%%%%%%%%%%%%%%%%%%%%%%%%%%%%%%%%%%%%%%%%%%%%%
\section{5. Zukunftsperspektiven}
%%%%%%%%%%%%%%%%%%%%%%%%%%%%%%%%%%%%%%%%%%%%%%%%%%%%%%%%%%%%%%%

\begin{frame}{Vom Analogen zum Digitalen}
  Moderne digitale Potentiometer (\emph{Digipots}) ersetzen mechanische Schleifer durch elektronische Schaltmatrizen.
  \begin{itemize}
    \item \textbf{Vorteile:} Keine mechanische Abnutzung, serielle Ansteuerung via I$^2$C/SPI.
    \item \textbf{Nachteile:} Begrenzte Auflösung (typ. 256 Stufen), eingeschränkte Spannungsbereiche.
  \end{itemize}
\end{frame}

\begin{frame}{Forschungsperspektive: Neue Materialien}
  \begin{itemize}
    \item \textbf{Graphen-basierte Leitbahnen}: extrem geringe Rauschwerte, hohe Belastbarkeit.
    \item \textbf{Nanostrukturierte Polymere}: flexible Potentiometer für Wearables.
    \item \textbf{Hybrid-Sensoren}: Kombination aus Potentiometer und Hall-Effekt für \emph{fail-safe}-Architekturen.
  \end{itemize}
\end{frame}




%%%%%%%%%%%%%%%%%%%%%%%%%%%%%%%%%%%%%%%%%%%%%%%%%%%%%%%%%%%%%%%
\section{6. Experimente: Mechanisch vs. Digital}
%%%%%%%%%%%%%%%%%%%%%%%%%%%%%%%%%%%%%%%%%%%%%%%%%%%%%%%%%%%%%%%

\begin{frame}{Linearität: Mechanisch vs. Digital}
\textbf{Ziel:} Vergleich der Ausgangsspannung \(U_{out}\) vs. Schleiferstellung / Schritte.

\begin{tikzpicture}[scale=1.0]
% Achsen
\draw[->, thick] (0,0) -- (6,0) node[right]{Schleiferposition / Schritt};
\draw[->, thick] (0,0) -- (0,4) node[above]{Ausgangsspannung $U_{out}$ (V)};

% Mechanisch (Drahtwicklung)
\draw[red, thick, smooth] plot coordinates {(0,0) (1,0.68) (2,1.42) (3,2.05) (4,2.75) (5,3.48)};
\node[red] at (4.5,3.2) {Mechanisch};

% Digital (256 Schritte)
\draw[orange, thick] plot coordinates {(0,0) (0.2,0.14) (0.4,0.28) (0.6,0.56) (0.8,0.7) 
                                      (1,0.98) (1.2,1.12) (1.4,1.4) (1.6,1.54) (1.8,1.82)
                                      (2,1.96) (2.2,2.24) (2.4,2.38) (2.6,2.66) (2.8,2.8)
                                      (3,3.08) (3.2,3.22) (3.4,3.5) (3.6,3.64) (3.8,3.92) (4,4)};
\node[orange] at (4.2,3.8) {Digital};
\end{tikzpicture}

\vspace{0.3cm}
\tiny{Mechanisch: leicht nichtlinear, Kontaktrauschen; Digital: diskret, verschleißfrei}
\end{frame}



% --------------------------
% Folie 1: Mechanische Potis
% --------------------------
\begin{frame}{Mechanische Potentiometer im Überblick}
\begin{columns}[T]
\begin{column}{0.6\textwidth}
\begin{block}{Drahtwickel-Poti}
\begin{itemize}
    \item Widerstand aus fein gewickeltem Draht
    \item Hohe Linearität und Genauigkeit
    \item Schleiferkontakt erzeugt Rauschen unter Last
    \item Ideal für präzise Messgeräte und industrielle Sensorik
\end{itemize}
\end{block}

\begin{block}{Leitplastik-Poti}
\begin{itemize}
    \item Widerstandsbahn aus leitfähigem Kunststoff
    \item Kostengünstig und langlebig bei geringer Belastung
    \item Leicht geringere Linearität und etwas Rauschen
    \item Typisch in Audio- und Konsumelektronik
\end{itemize}
\end{block}
\end{column}
\begin{column}{0.4\textwidth}
\includegraphics[width=\textwidth]{plastik.png} % Optionales Bild
\tiny{\textbf{Mechanische Potentiometer:} Drahtwickel-Poti links, Leitplastik-Poti rechts}
\end{column}
\end{columns}
\end{frame}



\begin{frame}{Rauschen & Belastungseffekt}
\textbf{Setup:} Schleifer mittlere Position, konstanter Strom, verschiedene Lastwiderstände $R_L$.

\begin{columns}[c]
\begin{column}{0.5\textwidth}
\begin{itemize}
    \item Drahtwicklung: CRV sichtbar, Spannungsschwankungen 50–100 mV
    \item Leitplastik: CRV kleiner, 20–50 mV
    \item Digital: praktisch null Rauschen, Lastunempfindlich
\end{itemize}
\end{column}
\begin{column}{4\textwidth}
\begin{tikzpicture}[scale=0.9] % scale reduziert, passt besser auf Folie
% Achsen
\draw[->, thick] (0,0) -- (6,0) node[below]{Zeit (s)};
\draw[->, thick] (0,0) -- (0,2) node[left]{Spannung (mV)};

% Drahtwicklung
\draw[red, thick, domain=0:5, samples=200, smooth] plot (\x, {1.6+0.3*sin(10*\x r)});

% Leitplastik
\draw[green, thick, domain=0:5, samples=200, smooth] plot (\x, {1.2+0.15*sin(8*\x r)});

% Digital
\draw[orange, thick, domain=0:5, samples=200, smooth] plot (\x, {0.7+0*sin(10*\x r)});

% Grid
\draw[gray!30, thin, step=0.5] (0,0) grid (6,2);

% Legende unterhalb, außerhalb des Diagramms
\node at (3, -0.8) {\small \textcolor{red}{Drahtwicklung} \quad \textcolor{green}{Leitplastik} \quad \textcolor{orange}{Digital}};

\end{tikzpicture}
\end{column}
\end{columns}
\end{frame}



\begin{frame}{Fazit: Mechanisch vs. Digital}
\begin{itemize}
    \item Digitalpotis: präzise Schritte, verschleißfrei, stabil bei Belastung
    \item Mechanische Potis: leicht nichtlinear, CRV, Verschleiß über Zeit
    \item Linearitätsmessung & Rauschmessung zeigen klar die Vorteile digitaler Lösungen
    \item Mechanisch weiterhin relevant für kostengünstige, einfache Anwendungen
\end{itemize}
\end{frame}



%%%%%%%%%%%%%%%%%%%%%%%%%%%%%%%%%%%%%%%%%%%%%%%%%%%%%%%%%%%%%%%
\section{7. Zusammenfassung}
%%%%%%%%%%%%%%%%%%%%%%%%%%%%%%%%%%%%%%%%%%%%%%%%%%%%%%%%%%%%%%%

\begin{frame}{Kernaussagen}
  \begin{enumerate}
    \item Potentiometer sind vielseitige, elektromechanische Transducer.
    \item Die Realität weicht vom idealen Spannungsteiler ab (Loading, CRV, parasitäre Effekte).
    \item Industrielle Anwendungen reichen von Audio bis Raumfahrt.
    \item Digitale und hybride Alternativen prägen die Zukunft.
  \end{enumerate}
\end{frame}

\begin{frame}[standout]
  Fragen?
\end{frame}

\end{document}
